\chapter{Types of Resolutions of Dominant Seventh Chords}

As we know, the power of the dominant chord comes from the resolution of its tritone to the following chord. Depending on the chord that follows the dominant we have three kinds of transitions of the tritone, which receive their name from the relative movements of the voices in the tritone.

\begin{enumerate}[label=\roman*), noitemsep]
	\item \textit{Contrary Movement}: Both notes of the tritone move in opposite directions.
	\item \textit{Oblique Movement}: One voice moves while the other keeps still.
	\item \textit{Parallel Movement}: Both notes move in the same direction.
\end{enumerate}

Some of these resolutions may appear in different kinds of resolutions. We provide all examples with a \chord{G}{7} as the dominant chord, remind the tritone of such dominant is found in the \ordinal{3}{rd} -- \ordinal{7}{th} $\equiv$ B -- F.

\begin{enumerate}[label=\roman*), noitemsep]
	\item \textit{Contrary Movement}: It is the strongest kind of resolution. We study four scenarios where this kind of resolution occurs.
	\renewcommand{\labelitemi}{$-$}
	\begin{itemize}
		\item Dominant resolves down a fifth to a I major chord. Such as \chord{G}{7} $\rightarrow$ \chord{C}{}, with the resolution being  B $\nearrow$ C $|$ F $\searrow$ E.
		
		\item Dominant resolves to the the minor relative (vi) of a I major chord (or also the \ordinal{1}{st} subtitute). Such as \chord{G}{7} $\rightarrow$ \chord{A}{m7}, with the resolution being  B $\nearrow$ C (why not A?) $|$ F $\searrow$ E.
		
		\item When the dominant V resolves to the second substitution of the minor i chord. For example, when the i chord is \chord{C}{m} then the \ordinal{2}{nd} substitution is \chord{A$\flat$}{maj7}. In this case the voice moves go as  B $\nearrow$ C $|$ F $\searrow$ E$\flat$.
		
		\item When we take the tritone subsitution of the dominant (subV) and resolve to the I major chord. In this case the tritone is inverted, and we have the same resolution as a (V-I): B $\nearrow$ C $|$ F $\searrow$ E; but the third of the subV resolves to the 3rd of I and the seventh of subV resolves to root of I (the opposite of a V-I).
	\end{itemize}
	
	\item \textit{Oblique Movement}: When the dominant V resolves to the second substitution of the I major chord. For example, when the I chord is \chord{C}{}, then the \ordinal{2}{nd} substitution is \chord{E}{min7}. In this case the voice moves go as  B = B $|$ F $\searrow$ E.
	
	\item \textit{Parallel Movement}: Dominant resolves to the the major relative (VI) of a i minor chord (or \ordinal{1}{st} subtitute). Such as \chord{G}{7} $\rightarrow$ \chord{E$\flat$}{maj7}, with the resolution being  B $\searrow$ B$\flat$ $|$  F $\searrow$ E$\flat$.
\end{enumerate}





