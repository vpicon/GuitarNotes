\chapter{CAGED System}
The CAGED system is the study of the chord shapes for C, A, G, E and D to play any chord on any fret in these forms. We give the forms for each of the chords:

\vspace{-10pt}
\begin{figure}[h]
	\centering
	\includegraphics[scale=1]{figs/caged/CAGED_forms.eps}
\end{figure} 
\vspace{-10pt}


Note that we have two E chord forms, one in the bass and another in the trebles, happening an octave above of the first one. These chords happen be divided into two kinds depending on the arrangement of chord tones in the chord shape: we have the [R, \ordinal{3}{rd}, \ordinal{5}{th}, R] kind (also [1358] for short) and the [R, \ordinal{5}{th}, R, \ordinal{3}{rd}] kind (or [1583]).

\begin{center}
	\def\arraystretch{1.3} % vertical padding in the table
	\begin{tabular}{| C{60pt} | C{60pt} |}
		\hline
		\textbf{[1358]} & \textbf{[1583]} \\ \hline
		G               & E               \\ \hline
		C               & A               \\ \hline
		E (trebles)        & D               \\ \hline
		
	\end{tabular}
\end{center}

We will study these two kind of chords as well as introduce other chord forms outside the CAGED system. 

\section{Chords [1583]}
As said before, we can move the forms of the CAGED system presented, to obtain other chords. Check the following examples obtained from the [1583] chord forms: 

\vspace{-10pt}
\begin{figure}[h]
	\centering
	\includegraphics[scale=1]{figs/caged/CAGED_1583_examples.eps}
\end{figure} 
\vspace{-10pt}

Most importantly, knowing where the chord degrees occur on the chord shape, we can obtain many other chords. Check for example the A form for many C seventh-type chords.

\vspace{-10pt}
\begin{figure}[h]
	\centering
	\includegraphics[scale=1]{figs/caged/CAGED_1583_Aforms.eps}
\end{figure} 
\vspace{-10pt}
%Figure C Cmin Cmaj7 C7 Cmin7 Cb5b7 Cdim





