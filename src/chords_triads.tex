\chapter{Chords: Triads and Seventh}
We begin studying chords with triads, in the same way as done in Walter Piston's tome. We continue with seventh chords, extremely used in modern harmony. 
\section{Triads}
The combination of two or more harmonic intervals makes up a \textit{chord}. The basic chord of common-practice harmony is the \textit{triad}, a group of three tones (called \textit{chord tones}) obtained by placing one third on top of another.

The names \textit{root} (R), \textit{third} (3) and \textit{fifth} (5) are given to the different tones of the triad, regardless of their arrangement. Any degree of the scale may serve as the root of a triad. For example, on the C major scale we obtain the following triads:

%Figure

As depicted above, the triads are denoted by the same roman numeral as their respective roots, but some notation is used depending on the kind triad occurring, as follows. The quality of the thirds conforming the triad, generate four kinds of triads, each of these having a different quality in their sound. 

{
\setlength{\tabcolsep}{1mm}
\def\arraystretch{1.5} % vertical padding in the table
\begin{tabular}{r p{12cm}  p{2cm}}
	i)   & A major third plus a minor third make a \textit{major} triad -denoted in capital roman letters, \chord{I}{}. & \quad Figure \\
	
	ii)  & A minor third plus a major third make a \textit{minor} triad -denoted in lowercase roman letters, \chord{ii}{}. & \quad Figure \\
	
	iii) & A minor third plus another minor third make a \textit{diminished} triad -denoted in lowercase roman letters, \chord{vi}{o}. & \quad Figure \\
	
	iv)  & A major third plus another major third make an \textit{augmented} triad -denoted in capital roman letters as \chord{III}{+}. & \quad Figure \\
\end{tabular}
}\\


In major and minor triads, it is the third which gives the chord its main flavor (major or minor), the fifth gives more saturation to the chord, but defines no quality. This is not true with diminished and augmented triads, where the fifth generally ads tension to the chord: look for example the diminished triad, where we have a tritone between the root and the fifth. \\

It is worth studying the kinds of triads that generate from the major scale (i.e. figure tal). Note that in the major scale there is augmented triad, this one will appear once we study the minor mode in further chapters.\\

\section{Seventh Chords}
Seventh chords are obtained appending another third interval on top of a triad. 
