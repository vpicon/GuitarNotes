\chapter{Chords: Triads and Seventh}
We begin studying chords with triads, in the same way as done in Walter Piston's tome. We continue with seventh chords, extremely used in modern harmony. 
\section{Triads}
The combination of two or more harmonic intervals makes up a \textit{chord}. The basic chord of common-practice harmony is the \textit{triad}, a group of three tones (called \textit{chord tones}) obtained by placing one third on top of another.

The names \textit{root} (R), \textit{third} (3) and \textit{fifth} (5) are given to the different tones of the triad, regardless of their arrangement. Any degree of the scale may serve as the root of a triad. For example, on the C major scale we obtain the following triads:

% Figure of the triads in the C major scale %

As depicted above, the triads are denoted by the same roman numeral as their respective roots, but some notation is used depending on the kind triad occurring, as follows. The quality of the thirds conforming the triad, generate four kinds of triads, each of these having a different quality in their sound. 

{
\setlength{\tabcolsep}{1mm}
\def\arraystretch{1.5} % vertical padding in the table
\begin{tabular}{r p{12cm}  p{2cm}}
	i)   & A major third plus a minor third make a \textit{major} triad -denoted in capital roman letters, \chord{I}{}. & \quad Figure \\
	
	ii)  & A minor third plus a major third make a \textit{minor} triad -denoted in lowercase roman letters, \chord{i}{}. & \quad Figure \\
	
	iii) & A minor third plus another minor third make a \textit{diminished} triad -denoted in lowercase roman letters, \chord{i}{o}. & \quad Figure \\
	
	iv)  & A major third plus another major third make an \textit{augmented} triad -denoted in capital roman letters as \chord{I}{+}. & \quad Figure \\
\end{tabular}
}\\


In major and minor triads, it is the third which gives the chord its main flavor (major or minor), the fifth gives more saturation to the chord, but defines no quality. This is not true with diminished and augmented triads, where the fifth generally ads tension to the chord: look for example the diminished triad, where we have a tritone between the root and the fifth. \\

It is worth studying the kinds of triads that generate from the major scale (i.e. figure tal). Note that in the major scale there is augmented triad, this one will appear once we study the minor mode in further chapters.\\

\section{Seventh Chords}
\textit{Seventh chords} are obtained appending another third interval on top of a triad, this tone is called the \textit{seventh} of the chord. As in triads, different combinations in the quality of the triads generate different chords, each with its own quality in their sound. We present the basic combinations of these, in a more schematic (to aid memorization), with all intervals referring to the root of the chord. \\

{
	\setlength{\tabcolsep}{1mm}
	\def\arraystretch{1.5} % vertical padding in the table
	\begin{tabular}{r c c c c c c p{2cm}}
		 \textit{Major Seventh} (\chord{I}{maj7}): \quad & Root & + & Maj. \ordinal{3}{rd} & + & Perf. \ordinal{5}{th} & + Maj. \ordinal{7}{th} & \quad Figure \\
		 
		 \textit{Dominant Seventh} (\chord{I}{7}): \quad & Root & + & Maj. \ordinal{3}{rd} & + & Perf. \ordinal{5}{th} & + min. \ordinal{7}{th} & \quad Figure \\
		 
		 \textit{Minor Seventh} (\chord{i}{min7}): \quad & Root & + & min. \ordinal{3}{rd} & + & Perf. \ordinal{5}{th} & + min. \ordinal{7}{th} & \quad Figure \\
		 
		 \textit{Half Diminished} (\chord{i}{\halfdim}): \quad & Root & + & min. \ordinal{3}{rd} & + & dim. \ordinal{5}{th} & + min. \ordinal{7}{th} & \quad Figure \\
		 
		 \textit{Diminished Seventh} (\chord{i}{o}): \quad & Root & + & min. \ordinal{3}{rd} & + & dim. \ordinal{5}{th} & + dim. \ordinal{7}{th} & \quad Figure \\
			
	\end{tabular}
}\\

First, the name is given, followed by the roman notation, then the intervals which form the chord and an example figure. There are much more seventh chords (such as minor major seventh, or augmented major seventh chords), just the common ones are shown in the above table.

In the same way as with triads, any degree of any (in our case diatonic) scale can serve as a root for a seventh chord. We give the resulting chords for the major scale, in particular the C major scale; the minor mode will be explained later. 

% Figure of the seventh chords obtained in the C major scale. %

It is worth learning the chords occurring in each degree of the scale. Note that the diminished seventh chord did not appear in the set of chords of the major scale, it will later appear in minor modes, and it is so special it deserves a chapter in its own. \\

\section{Chord Inversions}
Chords are defined just by the notes which conform them, whatever the arrangement. However, when we play a chord with its bass (the lower note) on a different note than the root, we refer to it as an \textit{inversion} of the chord. Other than the form with the root on the bass, we have three more inversions: with the bass on the \ordinal{3}{rd}, bass on \ordinal{5}{th} and bass on \ordinal{7}{th}, called \textit{first}, \textit{second} and \textit{third} inversions, respectively. Consider for example the different inversions of the \chord{C}{maj7} chord.

% Figure with the four forms of a Cmaj7 chord. %

Though all forms are inversions of the same chord, each one has a particular sound and quality which can be taken into account when playing them. Other than its quality, we may use them to write bass lines when concatenating several chords together. Chord inversions are better felt when playing them with the specific instrument.