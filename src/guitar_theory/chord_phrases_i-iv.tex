\chapter{Chord Phrases in i-iv Progression}
It is very common to have a i-iv cadence from tonic to subdominant chord, check for example a typical Tango chord progression:
\begin{center}
	\setlength{\tabcolsep}{1.3em}
	\def\arraystretch{1.5} % vertical padding in the table
	\begin{tabular}{c | c | c | c}
		
		i  & V & V & i \\ \hline
		iv & i & V & i \\
	\end{tabular}
\end{center}

\noindent In this framework we tend to make for interesting chord phrases to remark the subdominant chord (iv). We will study four typical forms for our i chord: E form, A form, D form and C form. The construction is given for the \chord{E}{m7}--$\,$\chord{A}{m7} cadence, but is completely similar for other forms. We begin with some basic inversions for a nice bass line.

\begin{center}
	\setlength{\tabcolsep}{1em}
	\def\arraystretch{1.5} % vertical padding in the table
	\begin{tabular}{c  c | c  c}
		
		\chord{E}{m7}  & \inversion{\chord{E}{m7}}{G} & 
		\chord{A}{m7}  & \inversion{\chord{A}{m7}}{C} \\

	\end{tabular}
\end{center}

\noindent Then we finally add secondary dominants to add some harmonic movement, also with the appropriate inversions to create a nice bass line.

\begin{center}
	\setlength{\tabcolsep}{1em}
	\def\arraystretch{1.5} % vertical padding in the table
	\begin{tabular}{c  c  c  c | c  c  c  c }
		
		\chord{E}{m7} 				 & \inversion{\chord{B}{7}}{F$\sharp$} & 
		\inversion{\chord{E}{m7}}{G} & \inversion{\chord{E}{7}}{G$\sharp$} & 
		\chord{A}{m7} 				 & \inversion{\chord{E}{7}}{B} &
		\inversion{\chord{A}{m7}}{C} & $\%$ \\
		
	\end{tabular}
\end{center}

\noindent The chord diagrams are given here for all forms.


\begin{figure}[h]
	\centering
	\includegraphics[width=\sheetfigsize]{figs/chord_phrases_i-iv/i-iv_progression_Em.eps}
	%\caption{}
	%\label{fig:1}
\end{figure} 